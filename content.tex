\section{基本}
    \subsection{型態大小}
        \documentclass{article}
\usepackage{amsmath}
\usepackage{geometry}

% Page layout
\geometry{a4paper, margin=1in}

\begin{document}
\title{Integer and Array Size Information}
\author{}
\date{}
\maketitle

\section*{Information}
\begin{itemize}
    \item \texttt{int}:
    \begin{itemize}
        \item Range: $-2,147,483,648$ to $2,147,483,647$ (10 digits)
        \item Using powers of 2: $-2^{31}$ to $2^{31} - 1$
        \item Using powers of 10: $-10^9$ to $10^9$
    \end{itemize}
    
    \item \texttt{unsigned long long int}:
    \begin{itemize}
        \item Begins with 9, and has a total of 19 digits
        \item Using powers of 2: $2^{63} - 1$
        \item Using powers of 10: $10^{18}$
    \end{itemize}
    
    \item \texttt{array}:
    \begin{itemize}
        \item Do not declare with a size larger than 30,005.
    \end{itemize}
\end{itemize}

\end{document}


        
\section{語法}
    \subsection{c++}
        \lstinputlisting{Contents/section1/basic.cpp}
    \subsection{c++函式庫}
        \lstinputlisting{Contents/section1/fun.cpp}
    \subsection{宣告法}
        \lstinputlisting{Contents/section1/declare.cpp}
    \subsection{強制轉型}
        \lstinputlisting{Contents/section1/turn.cpp}
    \subsection{python}
        \lstinputlisting{Contents/section1/tmp.py}
        
\section{字串}
    \subsection{KMP}
        \lstinputlisting{Contents/String/KMP.cpp}
    
\section{數論}
    \subsection{快速幕}
        \lstinputlisting{Contents/Math/fast.cpp}
    \subsection{窮舉(選or不選)}
        \lstinputlisting{Contents/Math/poor.cpp}
    \subsection{喵}
        \lstinputlisting{Contents/Math/01.cpp}
    \subsection{小費馬實踐}
        \lstinputlisting{Contents/Math/big_mod.cpp}
    \subsection{Fibonaccimal}
        \lstinputlisting{Contents/Math/Fibonaccimal.cpp}
    \subsection{LCM}
        \lstinputlisting{Contents/Math/LCM.cpp}
    \subsection{LCS}
        \lstinputlisting{Contents/Math/LCS.cpp}
    \subsection{LPS}
        \lstinputlisting{Contents/Math/LPS.cpp}
    \subsection{Pairty}
        \lstinputlisting{Contents/Math/Pairty.cpp}

\section{圖論}
    \subsection{最短路徑 dijkstra}
        \lstinputlisting{Contents/Graph/dijkstra.cpp}
    \subsection{DFS}
        \lstinputlisting{Contents/Graph/dfs.cpp}
    % \subsection{bellman-ford}
    %     \lstinputlisting{Contents/Graph/bellmanFord.cpp}
    \subsection{merge sort}
        \lstinputlisting{Contents/Graph/mergeSort.cpp}  
    \subsection{quick sort}
        \lstinputlisting{Contents/Graph/quickSort.cpp}
    \subsection{二分搜}
        \lstinputlisting{Contents/Graph/binarySearch.cpp}
\section{dp}
    \subsection{階乘1}
        \lstinputlisting{Contents/dp/p.py}
    \subsection{階乘2}
        \lstinputlisting{Contents/dp/a.cpp}
    \subsection{階梯問題}
        \lstinputlisting{Contents/dp/floor.cpp}
     \subsection{極值問題(格子有權重)}
        \lstinputlisting{Contents/dp/mew.cpp}
\section{數學}
    \subsection{理論}
        \begin{itemize}
  \item \textbf{三角形邊長定理}
  \begin{itemize}
    \item $a+b>c$
    \item 三角形形狀判定:
    \item \text{直角}$a^2+b^2=c^2$
    \item \text{鋭角}$a^2+b^2>c^2$
    \item \text{鈍角}$a^2+b^2<c^2$
  \end{itemize}
  
\end{itemize}

    \subsection{公式}
        \begin{itemize}
  \item \textbf{積}
  \begin{itemize}
    \item $\sum \limits_{i=1}^n i = \frac{n(n+1)}{2}$
    \item $\sum \limits_{i=1}^n i^2 = \frac{n(n+1)(2n+1)}{6}$
    \item $\sum \limits_{i=1}^n i^3 = \frac{(n^2(n+1)^2)}{4}$
    \item $\sum \limits_{i=1}^n i^4 = \frac{n(n+1)(2n+1)(3n^2+3n-1)}{30}$
    \item $\sum \limits_{i=1}^n i^5 = \frac{n^2(n+1)^2(2n^2+2n-1)}{12}$
    \item $\sum \limits_{i=1}^n i^6 = \frac{n(n+1)(2n+1)(3n^4+6n^3-3n+1)}{42}$
    \item $\sum \limits_{k=1}^n (k-1)(k-1)! = n!-1$
  \end{itemize}
  
  \item \textbf{log}
  \begin{itemize}
    \item $\log\frac{a}{b} = \log a - \log b$
    \item $\log_a b = \frac{\log a}{\log b}$
    \item $\log ab = \log a + \log b$
  \end{itemize}
  
  \item \textbf{三角形面積}
  \begin{itemize}
    \item $\sqrt{s(s-a)(s-b)(s-c)}, s = \frac{a+b+c}{2}$
  \end{itemize}
  
  \item \textbf{等差級數}
  \begin{itemize}
    \item $S_n = \frac{(a_1 + a_n)n}{2} = a_1n + \frac{(n-1)nd}{2}$
    \item $a_n = a_1 + (n-1)d$
  \end{itemize}
  
  \item \textbf{等比級數}
  \begin{itemize}
    \item $S_n = \frac{a_1(1-r^n)}{1-r} = \frac{a_1(r^n-1)}{r-1}, (r \neq 1)$
    \item $\frac{a_n}{a_{n-1}} = r$
  \end{itemize}
  
  \item \textbf{小費馬}
  \begin{itemize}
    \item $(a+b)\%n = (a\%n+b\%n)\%n$
    \item $(a*b)\%n = (a\%n*b\%n)\%n$
    \item $a*(b\%m) = (a*b)\%m$
  \end{itemize}
\end{itemize}

